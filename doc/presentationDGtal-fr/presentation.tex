\documentclass[8pt]{beamer}
%epackage[french]{babel}
\usepackage[latin1]{inputenc}
\usepackage{times}
\usepackage{wasysym}
\usepackage[T1]{fontenc}


\definecolor{mongris}{gray}{0.8}           % definition couleur grise
\newcommand{\dd}{\footnotesize $\Diamond$}

\newcommand{\HH}{ \vspace{0.5pt}\hrule}
\newcommand{\round}[1]{\lceil #1 \rfloor}  % notation arrondi
\def\eme{$^{\textrm{{\`e}me}}$}                  % i {\`e}me
\def\num{n^{\circ}}                        % numero
\def\Num{N^{\circ}}                        % Numero
\def\sinc{\mathrm{sinc}}                   % sinus cardinal
\def\ere{$^{\textrm{{\`e}re}}$}                % {\`e}re
\def\er{$^{\textrm{{e}r}}$}                % {\`e}re
\def\eg{\emph{e.g.} }                      % e.g.
\def\ie{\emph{i.e.} }                      % i.e.
\def\etc{\emph{etc}}                       % etc
\def\cm{\,cm}                              % cm
\def\met{\,m}                              % m
\def\mm{\,mm}                              % mm
\def\deg{$^\circ$}                         % degres
\def\ud{\mathrm{d}}                        % pour dx dy ...


\def \R {{\Bbb R}}
\def \I {{\Bbb I}}
\def \H{{\Bbb H}}
\def \F {{\Bbb F}}
\def \S {{\Bbb S}}
\def \B {{\Bbb B}}
\def \Z {{\mathbb Z}}
\def \G {{\mathbb G}}
\def \L {{\mathcal{L}}}
\def \C {{\mathcal C}}
\def \P {{\mathcal P}}
\def \Q {{\mathcal Q}} 
\def \E{{\mathcal E}}
\def \D{{\mathcal D}}
\definecolor{mybluecolor}{RGB}{116,121,149}

\newcommand{\darky}[1]{{\usebeamercolor[fg]{block title example} #1}}
\newcommand{\myblue}[1]{{\color{mybluecolor}\aut{[#1]}}}

\newcommand{\ball}  {\ensuremath{B}}
\newcommand{\AMDR}{\operatorname{AMD}}
\newcommand{\AMD}{\operatorname{AMD}}

\newcommand{\MAset}{\ensuremath{\mathrm{A\!M}} }
\newcommand{\MAsetg}{\ensuremath{\MAset^g } }

\def \PS {{\aut{Planar-4-3-SAT}}}
\def \R {{\Bbb R}}
\def \I {{\Bbb I}}
\def \F {{\Bbb F}}
\def \S {{\Bbb S}}
\def \Z {{\mathbb Z}}
\def \L {{\mathcal{L}}}
\def \C {{\mathcal C}}
\def \P {{\mathcal P}}
\def \Q {{\mathcal Q}} 
\def \E{{\mathcal E}}
\def \D{{\mathcal D}}
\def \BD {{\bar{\mathcal{D}}}}
\def \etal {{\it et al.~}}
\def\arc{\mbox{arc}}
\definecolor{mongris}{gray}{0.8}          
\newcommand{\fup}[1]{\uparrow#1\uparrow}
\newcommand{\fdown}[1]{\downarrow#1\downarrow}
\newcommand{\sI}[1]{\overline{\tt #1}}
\newcommand{\iI}[1]{\underline{\tt #1}}
\newcommand{\e}[5]{#1 & #2 & #3 & #4 & #5 \\}
\newcommand{\eh}[5]{\text{#1} & \text{#2} &  \text{#3} &  \text{#4} & \text{#5}\\} 

\usepackage{beamerthemeliris2}
\useoutertheme{smoothbars}

\title[]{DGtal: Digital  Geometry Tools and Algorithms Library}
\subtitle{\url{http://liris.cnrs.fr/dgtal}}

\author[DGtal~~~~~~~~~~~~~~~~~~~~~~~~~~David Coeurjolly]{David Coeurjolly}


 \newcommand{\fod}[2]{\multicolumn{2}{p{3.5cm}}{\emph{#1}\dotfill} &
      \multicolumn{2}{p{9cm}}{#2}\\}
    \newcommand{\fodt}[4]{\emph{#1} & {\footnotesize \textsl{#2}} & #3 & \small #4\\}
    % \newenvironment{ta}{\begin{tabular}{p{3.5cm}p{9cm}}}{\end{tabular}\\}
    \newenvironment{ta}{\begin{tabular}{crll}}{\end{tabular}\\}
    % \vfill


\newcommand{\aut}[1]{{\sc #1}}             % auteur en small capsu


\institute%[XXX]
{

  {\bf Laboratoire d'InfoRmatique en Image et Syst�mes d'information} \\
  { \scriptsize{
  LIRIS UMR 5205 CNRS/INSA de Lyon/Universit� Claude Bernard Lyon 1/Universit� Lumi�re Lyon 2/Ecole Centrale de Lyon\\
  INSA de Lyon, b�timent J. Verne\\
  20, Avenue Albert Einstein - 69622 Villeurbanne cedex\\
  \url{http://liris.cnrs.fr}}
  }
}



\graphicspath{{./Figures/}, {./Fig/}, {./ICPR2010/},{./Antoine/images/}}


\begin{document}

\small

\begin{frame}[plain]
  \titlepage
\end{frame}

\begin{frame}
\frametitle{Objectifs}

  \begin{block}{Biblioth�que G�om�trie discr�te\HH}
    \begin{itemize}
    \item faciliter l'appropriation de nos outils pour un n�ophyte
      (nouveau doctorant, chercheur d'une autre discipline, ...)
    \item tester rapidement une nouvelle id�e, permettre une meilleure
      comparaison d'un nouvel outil par rapport � l'existant
    \item faciliter la construction de d�monstrateurs (statiques, en
      ligne, ...)
    \item diffuser nos r�sultats de recherche � d'autres domaines
      \item mettre en place un projet f�d�rateur
\item \ldots
    \end{itemize}
  \end{block}
 

  \begin{block}{Qui ?\HH}
    \begin{itemize}
    \item LIRIS      
    \item LAMA (Chamb�ry)
    \item Gipsa-lab (Grenoble)
    \item LORIA (Nancy)
    \item GREYC (Caen)
    \end{itemize}
  \end{block}
\end{frame}

\begin{frame}
  \frametitle{Objectifs (bis)}

  
  \begin{block}{Pourquoi faire ?\HH}
    \begin{itemize}
    \item D�finir des objets discrets en dimension arbitraire
    \item Proposer des algorithmes d'analyse geom. et topo
    \item D�finir des m�canismes d'I/O et de visualisation
    \end{itemize}
  \end{block}
\end{frame}


\begin{frame}
  \frametitle{Structuration}

  \begin{block}{D�coupage\HH}
    \begin{itemize}
    \item Noyau topologique 
      \begin{itemize}
      \item Topologie digitale : ensembles, connexit�, bords,
        compos. connexes
      \item Topologie discr�te : mod�le cellulaire, contours,
        mod�le de repr�sentation de r�gions (Carte Topo/Combi)
      \item Calculs d'invariants n-D
      \end{itemize}
    \item Noyau g�om�trique
      \begin{itemize}
      \item Analyse contour : reconnaissance structures discr�tes,
        estimateurs g�om�triques, ...
      \item Analyse volumique : transformation en distance, axe
        m�dian, ...
      \end{itemize}
    \item Modules haut niveau
      \begin{itemize}
      \item \ldots        
      \end{itemize}
      
    \item Repr�sentation des images
      \begin{itemize}
      \item M�canisme de \emph{Container} g�n�rique
      \item Structures adapt�es pour de gros volumes (hashtree,...) 
      \end{itemize}
    \item Entr�es/sorties: 
      \begin{itemize}
      \item Visualisation vectorielle 2D (svg, xfig, eps)
      \item import/export  diff�rents formats d'image
      \item serialisation,...        
      \end{itemize}
    \end{itemize}
  \end{block}
\end{frame}

\begin{frame}
  \frametitle{M�thodologie}

  \begin{block}{Design\HH}
    \begin{itemize}
    \item C++
    \item Prog. g�n�rique
    \item Concepts et mod�les de concept
      \item LGPL ( ou \emph{GPL with restrictions})
    \end{itemize}
  \end{block}


  \begin{block}{Infrastructure\HH}
    \begin{itemize}
    \item  \url{http://liris.cnrs.fr/dgtal}
    \item Tests unitaires
    \item cmake/ctest/cdash
    \item Mailing lists
    \item Doxygen
    \item Tickets trac 
    \item multi-plateforme
    \end{itemize}
  \end{block}

\end{frame}

\begin{frame}

exemples...

\begin{itemize}
\item Space, Domain, Object
\item ImageContainer
\end{itemize}
\end{frame}

\end{document}



